%-----------------------------------------------------------------------------
%
%               Template for sigplanconf LaTeX Class
%
% Name:         sigplanconf-template.tex
%
% Purpose:      A template for sigplanconf.cls, which is a LaTeX 2e class
%               file for SIGPLAN conference proceedings.
%
% Guide:        Refer to "Author's Guide to the ACM SIGPLAN Class,"
%               sigplanconf-guide.pdf
%
% Author:       Paul C. Anagnostopoulos
%               Windfall Software
%               978 371-2316
%               paul@windfall.com
%
% Created:      15 February 2005
%
%-----------------------------------------------------------------------------


\documentclass{sigplanconf}

% The following \documentclass options may be useful:

% preprint      Remove this option only once the paper is in final form.
% 10pt          To set in 10-point type instead of 9-point.
% 11pt          To set in 11-point type instead of 9-point.
% authoryear    To obtain author/year citation style instead of numeric.

\usepackage{amsmath}


\begin{document}

\special{papersize=8.5in,11in}
\setlength{\pdfpageheight}{\paperheight}
\setlength{\pdfpagewidth}{\paperwidth}

\conferenceinfo{PLAS '14}{July 29, 2014, City, ST, Country}  
\copyrightyear{2014} 
\copyrightdata{978-1-nnnn-nnnn-n/yy/mm} % chktex 8
\doi{nnnnnnn.nnnnnnn}

% Uncomment one of the following two, if you are not going for the 
% traditional copyright transfer agreement.

%\exclusivelicense                % ACM gets exclusive license to publish, 
                                  % you retain copyright

%\permissiontopublish             % ACM gets nonexclusive license to publish
                                  % (paid open-access papers, 
                                  % short abstracts)

\titlebanner{banner above paper title}        % These are ignored unless
\preprintfooter{short description of paper}   % 'preprint' option specified.

\title{Position Paper: Breaking Barriers}
%\subtitle{Subtitle Text, if any}

\authorinfo{Name1}
           {Affiliation1}
           {Email1}
\authorinfo{Name2\and Name3}
           {Affiliation2/3}
           {Email2/3}

\maketitle

\begin{abstract}
This is the text of the abstract.
\end{abstract}

\category{CR-number}{subcategory}{third-level}

% general terms are not compulsory anymore, 
% you may leave them out
%\terms
%term1, term2

\keywords
keyword1, keyword2

\section{Introduction}\label{sec:intro}

In the past decade, there have been two converging trends in the design
of programming languages. The first trend is the relaxation of
information hiding mechanisms in languages: if we look at
object-oriented languages as an example, recent languages rarely enforce
property accessibility at the object level; instead, the acceptable
boundaries have increased to encompass modules or entire libraries. For
example, the Go programming language only enforces the public/private
distinction at the   package level, by controlling whether or not
symbols are exported. (Well, while this is true, it belies the fact that
Go is very strict about its exportability; it doesn't let you hack
around it using reflection.)

Definition of abstraction/encapsulation/information hiding (what are we talking about?)
	Enforces type safety, encapsulation, invariants

Sometimes, you want to violate abstraction:
	In a debugger (Why is something happening?)
	You want to change the behavior of a third-party library, without patching/forking it
	You need some extra performance, which can only be achieved by taking advantage of some specifics of the implementation of a library
	``Law of Leaky Abstractions''

This has been long recognized, in lots of different settings: software engineering, operating systems, programming languages, databases. The result is that there usually is a way to ``get what you want'' in real languages
However, sometimes, it is critical that abstraction is not violated, usually when security is concerned
	Running untrusted code, abstraction enforces security
	A compiler/runtime relies on abstraction being enforced in a very intricate way (e.g. very aggressive optimizations), and it's completely unreasonable for the programmer to be able to “guess” what behavior is going to occur

This is a relatively new development: everyone likes it when software runs faster, but people pay less attention when the abstraction violation leads to a security problems

PL theorists have mostly advocated to take the straight and narrow road, but this is not really tenable in reality [1]. On the other hand, taking the straight and narrow road is required if you want to give security guarantees.

The purpose of this paper is to look at this problem from a programming languages perspective

Argument: PL theory has already developed the conceptual framework for addressing these problems

Another angle: what features in a programming language do you need to have to internalize a debugger?
	Continuations
	But breakpoints? Need to have alternate behavior at “all steps” (that's expensive)

Angle: Take things people are doing, and then recast them in terms of the PL things that are going on

Angle: Abstraction is not binary

Angle: Design languages which accommodate existing social structures (organization into libraries, separate stakeholders, etc)

\section{Compiler correctness and gradual types}\label{sec:compiler}

Compilation from high-level semantics to low-level semantics

This is a classic topic which has received quite a bit of attention in the PL world, c.f. compiler correctness, FFI interoperability, and gradual typing

Schema: High level source language (abstraction barrier) Low level target language

Existing: Violate abstraction barrier by providing an FFI interface, for loading code compiled in the target language (actually, FFI tends not to expose the IR, but you could imagine the ability to directly load inside IR).

Characteristic of low-level language: untyped, encapsulation and invariants are not enforced

One approach: Typed assembly language, which brings static types to the low-level language, so that we can preserve guarantees (e.g. information-hiding from closures is preserved with an existential type constructor).

Another approach: Gradual typing, instead of preserving the invariant globally, maintain the invariants for statically typed code, allow them to be broken (type safety) in dynamically typed code, and use dynamically enforced contracts to ensure that statically typed programs can't ``go wrong''

OK, so what does this mean for normal programmers? Let's set up an analogy. The tower of intermediate languages now corresponds to the tower of abstractions presented to the user (alternately, library as a DSL). Normal clients of the library are writing code in the source language. Now, suppose I want to test some low-level feature of the language. I can either write on the top-level interface, trying to indirectly jiggle the part of the low-level API. Or I can break the abstraction barrier and directly interface with the low-level. That's like making an FFI call to the low-level language; or perhaps forgoing the high-level language entirely and working only in terms of the more ``internal'' API. Use of the internal API gives no guarantees that any implicit high-level invariants are respected. Then the TAL approach says, ``Well, maybe we can create a static type system which is strong enough to ensure that the use of the low-level API fulfills some sort of invariant'' whereas the gradual typing approach says, ``We can insert dynamic contracts (e.g. asserts) to ensure that when we transition to high-level code, the invariants they need fulfilled are good.''

\section{Monads, separation and effects}\label{sec:monads}

In Haskell, monads are used to separate effectful code from non-effectful code
Used in other settings too: in dependently typed languages, a monad can be used to separate partial programs from total programs (e.g. mtac)

Well known problem with monads in Haskell: if you have some pure code, which you now want to add effects to, you have to rewrite all your code in monadic style.
	DDC tries to address the version of this problem concerning writing code for mutable/immutable versions of a data structure. Introduce ``mutability polymorphism'', i.e. instead of IORef Int and Int, we have Int r (where r is the set of locations the object may lie)

	Monads


\section{Breakpoints and continuations}

Continuations are often considered hard to understand: one of my favorite explanations (Oleg's) utilizes debugging as a concrete example of continuations. A breakpoint is essentially jumping to the repl with the current continuation, which you can use to restart the computation.
	Actually, in a debugger, the continuations are used strictly linearly, so that you aren't saddled with an unsurmountable performance penalty

“Strong but unprincipled features”: eval, reflection, monkey-patching, meta-programming

\section{Proposals}

Programming languages that know about libraries and versions, and accommodate them explicitly
	Instead of multi-versions using preprocessing, make the multiple versions part of the AST, so (even if we can't compile it), we can typecheck all of the variants. (In that case, you write down the type-signature for the relied upon functions? That code doesn't change)
It should be OK to read private state, but not to write it.
	This preserves integrity, but not secrecy (does this mean you can do the dual?)
Discomfort when violating tenets of program design: seen in Haskell's unsafePerformIO and unsafeCoerce

\section{Examples}\label{sec:examples}

	Flag to override private modifier in C++
	Reflection in Java can be used to extract/edit values of private properties (security manager is responsible for disallowing such uses of reflection) (word on the street is that setAccessible is the ``only thing'' you have to turn off)
	Privacy in Python is completely optional and not enforced. (``Do not make the compiler/parser enforce privacy since developers who want to access private members have ways to do so anyway.'')
	Haskell libraries often have C preprocessor, to vary across versions (when an API changes is similar to the situation which occurs when an internal implementation detail you were relying on changes). So even with perfect encapsulation, if you accept APIs change, you have to deal with this!
	Firefox/Chrome IDL for defining interfaces; you can declare some “chrome-only” (available only to trusted code). Similar situation for interplay between JavaScript of extensions and sites: a malicious site can hijack a buggy extension's JavaScript, so a boundary here is important
	Linux syscall interface. The syscall interface is ostensibly an abstraction boundary, but it is constantly expanding in size, because developers demand fine-grained control from the kernel (consider the ioctl command). However, when the syscall API is extremely huge, it's hard to ensure it's secure; for example, the ptrace syscall was restricted in order to make Linux malware harder to write
	Modern programming languages have moved away from per-class access modifiers, moving them to operate at a library or module level
	Haskell compiler pragmas: some offer the ability to access internal identifiers (such as MagicHash); so use of the pragma indicates you are willing to use internal implementation details
	Mazieres: Emacs as an example of a large system, without encapsulation, which works well. ``UNIX Emacs: A Retrospective'' fingers some particular features: actually, Emacs lisp is quite abstract with respect to its core data structures, the text buffer, word, save states. With simple operations on them which reflect normal Emacs use (if you can do it interactively, you can script it as well). Empirical question: Emacs plugin dependency (answer: lots of emacs plugins have no dependencies, but there is a central tangle of a number of plugins, and the chains can be pretty big)
	Mazieres: LaTeX has an internal implementation of floating point because there was not a way to access native floating point support in the hardware
	Mazieres: Databases and btrees; SQL is an example where the underyling data structures are abstracted away, but otherwise, there is no encapsulation (the SQL language coincides with the system administration language)
	Log-based file system: if I'm building a data structure whose form of persistence is a log, I shouldn't have to log again in the filesystem.
	Deian: Rethink the sync, the IDs that I get should be usable in the app, even if not modifiable


%   \acks

%   Acknowledgments, if needed.

% We recommend abbrvnat bibliography style.

\bibliographystyle{abbrvnat}

% The bibliography should be embedded for final submission.

%   \begin{thebibliography}{}
%   \softraggedright

%   \bibitem[Smith et~al.(2009)Smith, Jones]{smith02}
%   P. Q. Smith, and X. Y. Jones. ...reference text...

%   \end{thebibliography}


\end{document}

%                       Revision History
%                       -------- -------
%  Date         Person  Ver.    Change
%  ----         ------  ----    ------

%  2013.06.29   TU      0.1--4  comments on permission/copyright notices

